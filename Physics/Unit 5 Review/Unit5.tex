\documentclass{article}
\usepackage{amsmath, amssymb}

\title{Comprehensive Guide to Electricity and Magnetism}
\author{}
\date{}

\begin{document}
\maketitle

\section{Introduction}
Electricity and magnetism are fundamental aspects of physics that describe the interaction of electrical charges through electric and magnetic fields. This text explores the laws, principles, and mathematical formulations that govern these interactions.

\section{Electrostatics}
\subsection{Electric Charge and Coulomb's Law}
Electric charges are the source of electric fields. They come in two types: positive and negative. Coulomb's Law quantifies the force between two point charges:
\begin{equation}
    F = k \frac{|q_1 q_2|}{r^2}
\end{equation}
where \( F \) is the magnitude of the force between the charges, \( q_1 \) and \( q_2 \) are the charges, \( r \) is the distance between the charges, and \( k \) is Coulomb's constant.

\subsection{Electric Field and Electric Potential}
The electric field \( \vec{E} \) created by a charge \( q \) at a distance \( r \) is given by:
\begin{equation}
    \vec{E} = k \frac{q}{r^2} \hat{r}
\end{equation}
where \( \hat{r} \) is the unit vector pointing away from the charge. The electric potential \( V \), related to the electric field, is defined as:
\begin{equation}
    V = -\int \vec{E} \cdot d\vec{l}
\end{equation}

\end{document}
